Emitting the code is the simplest of all phases.
Since Parrot requires all the variables to be declared,
a comma separated string \verb|$dec|  is built
concatenating the keys of the symbol table hash \verb|%s|.
The code is then indented and the different components
are articulated through a HERE document:
\begin{verbatim}
    sub output_code {
      my ($trans, $dec) = @_;

      # Indent
      $$trans =~ s/^/\t/gm;

      # Output the code
    print << "TRANSLATION";
    .sub 'main' :main
    \t.local num $$dec
    $$trans
    .end
    TRANSLATION
\end{verbatim}
The call to \verb|output_code| finishes the job:
\begin{verbatim}
    output_code(\$t->{trans}, \$dec);
\end{verbatim}

The translation is approached as a particular case of 
\I{tree decoration}. Each node is decorated with a new
attribute - \verb|trans| - that will held 
the translation for such node.
To compute it, we must define transformations
for each of the types in the AST:
\begin{verbatim}
translation = t_num t_var t_op t_neg
             t_assign t_list t_print;
\end{verbatim}

Some of these transformations are straightforward:
\begin{verbatim}
t_num: NUM 
  => { $NUM->{tr} = $NUM->{attr} }
t_op:  /TIMES|PLUS|DIV|MINUS/:b($x, $y)
  => {
    my $op = $Op{ref($b)};
    $b->{tr} = "$b->{reg} = $x->{reg}"
                   ." $op $y->{reg}";
  }
\end{verbatim}
To keep track of the involved variables
a hash is used as a rudimentary symbol table:
\begin{verbatim}
{ our %s; }
t_assign: ASSIGN($v, $e) => {
  $s{$v->{attr}} = "num";
  $ASSIGN->{tr} = "$v->{reg} = $e->{reg}"
}
\end{verbatim}
The translation of the root node (\verb|EXPS|)
consists of concatenating the translations
of its children:
\begin{verbatim}
{
  sub cat_trans {
    my $t = shift;

    my $tr = "";
    for ($t->children) {
      (ref($_) =~ m{NUM|VAR|TERMINAL})
        or $tr .= cat_trans($_)."\n"
    }
    $tr .= $t->{tr} ;
  }
}

t_list: EXPS(@S)
  => {
    $EXPS->{tr} = "";
    my @tr = map { cat_trans($_) } @S;
    $EXPS->{tr} =
      reduce { "$a\n$b" } @tr if @tr;
  }
\end{verbatim}
The treeregexp \verb|@S| matches the children
of the \verb|EXPS| node. The associated lexical variable \verb|@S| 
contains the references to the nodes that 
matched.

The method \verb|bud|\footnote{\underline{B}ottom-\underline{U}p \underline{D}ecorator}
of \verb|Parse::Eyapp::Node| nodes makes a bootom up traversing
of the AST applying to the node being visited the only one transformation that 
matches\footnote{When {\tt bud} is applied the family of transformations \underline{must} constitute
a \I{partition} of the AST classes, i.e. for each node one and only one
transformation matches}.
After the call 
\begin{verbatim}
$t->bud(our @translation);
\end{verbatim}
the attribute \verb|$t->{trans}| contains 
a translation to PIR for the whole tree.

Almost any Perl programmer knows what {\it Parsing} is about.
One of the strengths of Perl is its excellence for text analysis. 
Additionally to its embedded regular expression capacities, modules 
like \code{Parse::RecDescent} \cite{conwayrd} and 
\code{Parse::Yapp} \cite{desarmenien} make easier the task of text 
understanding and text transformation.
%This is in clear contrast with the absence of Perl 5 
%generic tools\footnote{There are however very good specific ones,
%for example, for \code{XML} and \code{HTML} support and 
%symbolic mathematics
% \cite{muller}
%}
%giving support for the subsequent stages
%of text processing.
%The exception being the module \code{Language::AttributeGrammar}
%\cite{luke}. Parrot does well in this 
%chapter, having the Parrot Grammar Engine (PGE) \cite{michaud}
%and the Tree Grammar Engine (TGE)
%\cite{randaltge}.

\code{Parse::Eyapp} (Extended yapp) is a collection of modules
that extends Francois Desarmenien \code{Parse::Yapp} \code{1.05}:
Any \code{yapp} program runs without changes with \code{eyapp}.
Additionally \code{Parse::Eyapp} provides new 
functionalities like named attributes,
EBNF-like expressions, modifiable default actions,
abstract syntax tree building and translation schemes. 
It also provides a language for tree transformations.

This article introduces the basics of 
translator construction with \code{Parse::Eyapp} through an
example that compiles infix expressions into Parrot 
Intermediate Representation (PIR)\cite{randal04}.
Parrot is a virtual machine (VM), similar to the Java VM and the
.NET VM. However, unlike these two which are designed for
statically-typed languages like Java or C\#, Parrot is designed for
use with dynamically typed languages such as Perl, Python, Ruby,
or PHP.

The input to the program will be a (semicolon separated)
list of infix expressions, like:

\begin{verbatim}
1  b = 5;
2  a = b+2;
3  a = 2*(a+b)*(2-4/2); # is zero
4  print a;
5  d = (a = a+1)*4-b;
6  c = a*b+d;
7  print c;
8  print d
\end{verbatim}

and the output is an equivalent
PIR: 
\begin{verbatim}
 1  .sub 'main' :main
 2     .local num a, b, c, d
 3     b = 5
 4     a = b + 2
 5     a = 0 # expression at line 3 
 6     print "a = "     # above was
 7     print a    # reduced to zero
 8     print "\n" # at compile time
 9     a = a + 1
10     $N5 = a * 4
11     d = $N5 - b
12     $N7 = a * b
13     c = $N7 + d
14     print "c = "
15     print c
16     print "\n"
17     print "d = "
18     print d
19     print "\n"
20  .end
\end{verbatim}

The Parrot VM is register based. This means that, like a hardware
CPU, it has a number of fast-access units of storage called registers.
There are 4 types of register in Parrot: integers (I), numbers (N),
strings (S) and PMCs (P). For each type there are several of these, named
$N0,$N1, \ldots , Number registers map
to the machine native floating point type. 

The code above is an example of PIR, which stands for 
Parrot Intermediate Representation and is also known as
Intermediate Code or IMC. PIR files use the extension .pir.
PIR is an intermediate language that can be compiled to
Parrot Byte code. It was conceived as a possible target language for compilers
targeting the Parrot Virtual Machine. PIR is halfway between a High
Level Language (HLL) and Parrot Assembly (PASM).

You can download the code for this example from
\htmladdnormallink
{http://nereida.deioc.ull.es/~pl/eyapsimple/source.tgz}
{http://nereida.deioc.ull.es/~pl/eyapsimple/source.tgz}.
To use it, unpack the tarball:
\begin{verbatim}
tar xvzf source.tgz
\end{verbatim}
Change to the directory:
\begin{verbatim}
cd src
\end{verbatim}
and compile the grammar with \verb|eyapp|:
\begin{verbatim}
eyapp Infix.eyp
\end{verbatim}
Compile also the set of tree transformations using \verb|treereg|:
\begin{verbatim}
treereg -m main I2PIR.trg
\end{verbatim}
After these two compilations we have two new modules:
\begin{verbatim}
nereida:/tmp/src> ls -ltr |tail -2
-rw-rw----  1 pl users   Infix.pm
-rw-rw----  1 pl users   I2PIR.pm
\end{verbatim}
Module \verb|Infix.pm| contains the parser for the grammar described in \verb|Infix.eyp|.
Module \verb|I2PIR.pm| contains the collection of tree transformations described
in \verb|I2PIR.trg|. Now we can run the script \verb|infix2pir.pl| which makes use of these
two modules:
\begin{verbatim}
$ ./infix2pir.pl input1.inf > input1.pir
\end{verbatim}
We can now make use of the \verb|parrot| interpreter to execute the code:
\begin{verbatim}
$ /Users/casianorodriguezleon/src/parrot/parrot-1.9.0/parrot input1.pir 
a = 0
c = 4
d = -1
\end{verbatim}

See \htmladdnormallink{http://www.parrot.org/download}{http://www.parrot.org/download} 
for several ways to get a recent version of parrot.

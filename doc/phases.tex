The code below 
(file \verb|src/infix2pir.pl|) %
displays
the stages of the translator: \emph{Lexical
and syntax analysis, tree transformations
and decorations,
address assignments, code generation
and peephole optimization}. The simplicity of the 
considered language (no types, no control structures) 
permits the skipping of 
\emph{context handling} (also called \emph{semantic analysis}).
Context handling includes jobs like \emph{type checking}, 
\emph{live analysis}, etc.
Don't get overflowed for so much terminology:
The incoming sections will
explain in more detail each of these phases.
\begin{verbatim}
my $parser = Infix->new();
# Set input 
$parser->YYData->{INPUT} 
  = slurp_file($filename, 'inf');

# Lexical and syntax analysis
my $t = $parser->YYParse(
 yylex => \&Infix::Lex, 
 yyerror => \&Infix::Err);

# Tree transformations
$t->s(our @algebra);

# Address assignment
our $reg_assign;
$reg_assign->s($t);

# Code generation
$t->bud(our @translation);
my $dec = build_dec();

peephole_optimization($t->{tr});

output_code(\$t->{tr}, \$dec);
\end{verbatim}

The compiler uses the parser for infix expressions
that was generated from the Eyapp grammar  \verb|Infix.eyp|
(see section \ref{section:syntacticanalysis})
using the command:
\begin{verbatim}
$ eyapp Infix.eyp
$ ls -tr | tail -1
Infix.pm
\end{verbatim}
It also uses the module containing different families of tree
transformations that are described in the \verb|I2PIR.trg| file 
(explained in 
sections \ref{section:machineindependentoptimizations} and \ref{section:codegeneration}): 
\begin{verbatim}
$ treereg -m main I2PIR.trg
$ ls -tr | tail -1
I2PIR.pm
$ head -1 I2PIR.pm
package main;
\end{verbatim}
The option \verb|-m main| tells \verb|treereg|
to place the transformations inside the \verb|main|
namespace. 
% Make a makefile! or make a dist with the example and put it in CPAN
